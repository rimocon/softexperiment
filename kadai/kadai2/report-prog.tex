\documentclass{jarticle}
\usepackage[dvipdfmx]{graphicx}

\title{ソフトウェア設計及び実験\\}
\author{6119019056 山口力也}
\date{2019/04/23日提出}


\begin{document}
\maketitle
\section{プログラミング作法について}

私は高専時代に卒業研究で深層学習について研究していた.その時初めてPythonを学び,プログラミング作法の大切さを痛感した.Pythonを学ぶ前もある程度インデントくらいはしていたが,編集している途中にずれることがよくあった.しかし,それまでに触ったCやJava,Fortranといった言語ではインデントがずれていてもプログラム自体は動いてしまっていたので,特に気にすることもなかった.Pythonではインデントが統一されており,私のように変なインデントをしているとプログラムが動かない.そのため卒業研究の序盤はそこでかなり苦しみ,同期のみなが研究に勤しんでいる中自分はプログラミングの基礎中の基礎のインデントをもう一度復讐していた.結果的にはインデントはvimのコマンドで"gg=G"と打ってしまえばほぼすべて解決したが,インデントに限らず変数の定義の仕方など,担当教員から何度も注意され,そこでプログラミング作法の大切さに気付いたと思う.しかし,=の前にスペースを入れるべきか否か,if文の"\{"は改行するべきか否か,など人によって意見が異なるプログラミング作法もあると思う.実際,自分の周りでも意見が割れていた.そういったものに関してはその人の"流儀"として受け入れるべきなのかもしれない.みながみな同じようにコードを書いていれば理解しやすく効率的だが,現実問題それは不可能だ.もちろんプログラミング作法に忠実であることは大切であるが,絶対にこの書き方じゃないとダメ!ではなく,他の人の書き方も受け入れる柔軟性も必要だと思う.長々と書いてしまったが,結局言いたいのはプログラミング作法について学ぶことは重要であるということである.プログラミング作法を学べば,単にコードが綺麗になるだけでなく,様々なことが学べると思う.自分は適当なエディタで適当にTabキーを押していたためインデントに苦しんだが,その解決策としてvimという最高のテキストエディタに巡り会えた.他の人だったらもっと良い発見があるかもしれない.
とは言っても,自分はまだまだ初心者で他の人のコードを見ると自分のコードの汚さと無駄の多さを感じる.これからもプログラミング作法に注意を払い,グループ開発などで誰が見ても理解できるようなわかりやすいコーディングを心掛けたい.
\end{document}
