\documentclass{jarticle}
\usepackage[dvipdfmx]{graphicx}

\title{ソフトウェア設計及び実験\\}
\author{6119019056 山口力也}
\date{2019/04/16日提出}


\begin{document}
\maketitle
\section{プログラミング作法について}
私は高専時代に卒業研究で深層学習について研究していた.その時初めてPythonを学び,プログラミング作法の大切さを痛感した.Pythonを学ぶ前もある程度インデントくらいはしていたが,編集している途中にずれることがよくあった.しかし,それまでに触ったCやJava,Fortranといった言語ではインデントがずれていてもプログラム自体は動いてしまっていたので,特に気にすることもなかった.Pythonではインデントが統一されており,私のように変なインデントをすると動かない.そのため卒業研究の序盤はそこでかなり苦しみ,同期のみなが研究に勤しんでいる中自分はプログラミングの基礎中の基礎のインデントをもう一度復讐していた.結果的にはインデントはvimのコマンドで"gg=G"と打ってしまえばほぼすべて解決したが,インデントに限らず変数の定義の仕方など,担当教員に指導してもらったおかげでプログラミング作法の重要さを理解できたと思う.なによりそれまで適当なテキストエディタで適当にTabキーを押していた自分が,vimという最高のテキストエディタに巡り会えたという意味では良い機会だったように思う.

自分はまだまだ初心者で他の人のコードを見ると自分のコードの汚さと無駄の多さを感じる.これからもプログラミング作法に注意を払い,グループ開発などで誰が見ても理解できるようなわかりやすいコーディングを心掛けたい.

\end{document}
