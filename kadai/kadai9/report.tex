\documentclass{jarticle}
\usepackage[dvipdfmx]{graphicx}
\usepackage{here}
\usepackage{listings,jlisting} %日本語のコメントアウトをする場合jlistingが必要
%ここからソースコードの表示に関する設定
\lstset{
	basicstyle={\ttfamily},
		identifierstyle={\small},
		commentstyle={\smallitshape},
		keywordstyle={\small\bfseries},
		ndkeywordstyle={\small},
		stringstyle={\small\ttfamily},
		frame={tb},
		breaklines=true,
		columns=[l]{fullflexible},
		numbers=left,
		xrightmargin=0zw,
		xleftmargin=3zw,
		numberstyle={\scriptsize},
		stepnumber=1,
		numbersep=1zw,
		lineskip=-0.5ex
}
\title{ソフトウェア設計及び実験\\
	第9回レポート}
\author{6119019056 山口力也}
\date{2019/06/18日提出}


\begin{document}
\maketitle
\section{テストログ}
以下表\ref{table:testlog}にテストログの表を示す.

\begin{table}[H]
  \centering
  \caption{テストログ}
  \begin{tabular}{|c|p{2cm}|p{2cm}|p{2cm}|p{2cm}|} \hline
    No. & 手順 & 期待する結果 & 判定 & 不具合No. \\ \hline
	 1 & マップ情報ファイルが存在しない状態でプログラムを起動する & エラーメッセ-ジを表示して終了する & $\circ$ & \\ \hline
    2 & プレイヤー動作を確認 & キャラクターが正常に表示される & $\times$ &1 \\ \hline
	 3 & プレイヤー動作を確認 & 穴に落ちるとGAMEOVERになる & $\times$ & 2 \\ \hline
	 4 & ボール動作を確認 & スペースキーを押すと横方向にボールが飛ぶ & $\times$ & 3 \\ \hline
	 5 & 敵の動作を確認 & 拠点が描画される & $\times$ & 4 \\ \hline
	 6 & ボスの動作を確認 & 拠点を3つ倒すとボスが出現 & $\times$ & 5 \\ \hline 
	 7 & 敵の動作を確認 & プレイヤーと敵が接触するとhpが減る & $\times$ & 6 \\ \hline
  \end{tabular}
  \label{table:testlog}
\end{table}

\section{不具合報告書}
以下表\ref{table:huguai}に不具合報告書の表を示す.
\begin{table}[H]
  \centering
  \caption{不具合報告書}
  \begin{tabular}{|c|p{2cm}|p{2cm}|p{2cm}|p{2cm}|p{2cm}|} \hline
    No. & タイトル & テストNo. & 再現率 & 原因 & 修正内容 \\ \hline
	 1 & 起動時にキャラが上に飛んでいく & 1 & 毎回 & y方向の重力と速度の式が間違っていた& newvely.yとnewpoint.yの式を変更した  \\ \hline
    2 & 穴に入るとプログラムが落ちる & 2 & 毎回 & 画面外(下方向)での処理が記述されていなかった &下方向の画面外にキャラが移動したときの処理を記述した   \\ \hline
	 3 & ボールが横に飛ばない & 3 & 毎回 & ボールに横方向の速度が与えられていなかった & MoveChara関数のボールの処理に横方向の速度を記述 \\ \hline
	 4 & 拠点が瞬間移動する & 4 & 毎回 & 拠点を固定する記述がない & MoveChara関数内で拠点の場合何もしない記述を追加した  \\ \hline
	 5 &  拠点を3つ倒してもボスが出現しない & 5 & 毎回 & 拠点を何個倒したか保存できていない & 拠点を3つ倒した時にWarningがでる処理を記述 \\ \hline
	 6 &画面左側から出現した敵に当たり判定がない & 6 & 毎回 & わからなかった &   \\ \hline 
  \end{tabular}
  \label{table:huguai}
\end{table}

\end{document}
